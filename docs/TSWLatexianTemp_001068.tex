\documentclass[11pt]{revtex4}

\usepackage[utf8x]{inputenc}
\usepackage{graphicx}
\usepackage{amssymb}
\usepackage{epstopdf}
\usepackage{mystyle}
\usepackage{amsmath}
\usepackage{bm}

\begin{document}


\title{Correlation function}
\author{Bing Gu}
\date{8/4/14}
\maketitle

\section{QM correlation}
Quantum mechanical correlation is defined as 
\be C_{AB} = Tr[\hat{\rho} AU^\dagger(t)BU(t) ] \ee 
where $U(t)$ is the propagation operator for time-independent Hamiltonian, 
\be U(t) = \exp(-\im \hat{H}t/\hbar) \ee   
At absolute zero, 
\be \hat{\rho} = | \psi_0 \ket \bra \psi_0 |, \ee
then correlation becomes 
\begin{flalign}
 C_{AB} & = \bra \psi_0 | \hat{A}U^\dagger(t)\hat{B}U(t) | \psi_0 \ket  \\ 
	     & =    \bra \psi_0 | AU^\dagger(t)B | \psi_t \ket \\ 
	     & = \bra U(t)A^\dagger\psi_0 | B | \psi_t \label{eq:corr} \ket  
\end{flalign} 
Assume $\hat{A} = \sum_i \hat{x}_i$ and define $\phi(\bm x,0) = \sum_i x_i\psi(\bm x,0)$, $\phi(\bm x,t)$ is a wavefunction with nodes, at least one node at $\bm x=0$ at starting point, then $  U(t)A^\dagger\psi_0 $ is equivalent to propagate wavefunction $\phi(\bm x,t)$.

A wavefunction with nodes can be described by product of a nodeless wavepacket and a polynomial function  
\be \phi(\bm x,t) = \psi(\bm x,t) \chi(\bm x,t), \label{eq:mix} \ee
where $\chi(\bm x,t)$ is represented by a polynomial basis $f(\bm x)$, 
\be \chi(\bm x,t) = \bm f^T \bm c = \sum_i f_ic_i \ee 
For linear basis, $f(\bm x) = (1,x_1,x_2,\dots, x_{N_{dim}})$, $N_{dim}$ is the number of degree of freedom, usually taken as $3 \times N_{atom}$ in Cartesian space.
Substitute Eq. (\ref{eq:mix}) into time-dependent \se, after some algebraic manipulations, one obtain 
\be \dot{\bm c} = - \frac{\hbar}{2m}\bf{M}^{-1}(2\Pi + \im \bf  D)\bm c \label{eq:prop_c}. \ee
where 
\begin{flalign}
 \Pi_{ij} & = \sum_{\alpha=1}^{N_{dim}} \bra p_\alpha f_i | \grad_\alpha f_j \ket, \\ 
D_{ij} & = \sum_{\alpha=1}^{N_{dim}} \bra \grad_\alpha f_i | \grad_\alpha f_j \ket \\ 
M_{ij} & = \bra f_i | f_j \ket 
\end{flalign}   
Here $\bra \cdots \ket$ represents average over ensemble of quantum trajectories. 
Substitute Eq. (\ref{eq:mix}) back into Eq. (\ref{eq:corr}), we obtain 
\be C_{AB}(t) = \bra \psi(\bm x,t)\chi(\bm x,t) | \hat{B} | \psi(\bm x,t) \ket \ee
Assume $\hat{B}$ is a multiplicative operator, most commonly is position-dependent operators, $\hat{B}(\bm x)$,  
\begin{flalign}
 C_{AB}(t) &= \int d\bm x \rho(\bm x,t) \chi^*(\bm x,t)B(\bm x) \\ 
		 &\approx  \int d\bm x \rho(\bm x,t) [\bm c^*(t) \cdot \bm f(\bm x)] B(\bm x) \\ 
		 & = \sum_{k=1}^{N_{traj}} w^{(k)} [\bm c^*(t) \cdot \bm f(\bm x^{(k)})] B(\bm x^{(k)})
\end{flalign}

  
\end{document}